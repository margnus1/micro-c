\documentclass[a4paper,11pt]{article}

%% We can use macros to avoid typing the same thing over and over
\newcommand{\token}[1]{\texttt{<#1>}}
\newcommand{\uC}{{$\mathrm{\mu}$}C }

\title{Assignment 1: Lexical Analyzer}
\author{Xiao Yang \and Magnus L{\aa}ng} % replace by your name(s)
\date{\today}
\begin{document}
\maketitle
\section{General Description}
We use the JavaCC Framework to implement a lexical analyzer for the \uC
language. All tokens that need to be detected are specified in the
\texttt{lexer-test.jj} file.

\section{Technical Issues}
\subsection{Comments}
\begin{description}
	\item[Single Line Comment] The token is specified to start with ``\textbf{//}'' and end at end-of-line (``\textit{\textbackslash n}'', ``\textit{\textbackslash r}'' or ``\textit{\textbackslash r\textbackslash n}'').
	\item[Multiline Comment] We use keyword \emph{MORE} to specify the rule when we see \textit{/*} : change from state \emph{DEFAULT} to state \emph{IN\_MULTI\_LINE\_COMMENT}. Within \emph{IN\_MULTI\_LINE\_COMMENT}, any character will be matched until \textit{*/} is seen. So the whole comment will be recognized as one token and the lexical state will be changed back to \emph{DEFAULT} afterwards.

\end{description}
\subsection{End-of-file}
When JavaCC encounters end-of-file, it tries to end the current token, and then
inserts a last \token{EOF} token in the token stream. If no token type matches
the lexeme at that point, it will generate a lexical error.

If there is an unterminated multi-line comment, there will be no rule matching
that lexeme and JavaCC will generate a lexical error. However, if end-of-file is
found inside a single-line comment, that lexeme will be accepted as a
single-line comment, since end-of-line is not required at the end of the single
line comment special token.

\subsection{Source Code Position}
Source code position information is provided automatically by JavaCC in the
\texttt{Token} base class.

\end{document}
